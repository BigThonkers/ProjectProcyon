% !TeX spellcheck = de_DE
\PassOptionsToPackage{dvipsnames}{xcolor}
\documentclass[titlepage,11pt,a4paper,ngerman]{report}
\usepackage[utf8]{inputenc}
\usepackage[T1]{fontenc}
\usepackage[german]{babel}
\usepackage{graphicx}
\usepackage{wrapfig}
\usepackage{amsmath}
\usepackage{amsfonts}
\usepackage{amssymb}
\usepackage{tikz}
\usepackage{tikz-cd}
\usepackage{nicefrac}
\usepackage{mathtools}
\usepackage{enumerate}
\usepackage{cancel}
\usepackage[hidelinks]{hyperref}
\usepackage{cleveref}
\usepackage{tcolorbox}
\usepackage{enumitem}
\usepackage[margin=1in]{geometry}
\usepackage{placeins}
\usepackage{booktabs}
\usepackage{url}
\usetikzlibrary{arrows}
\usepackage[toc,page]{appendix}
\usepackage{bm}

%Change name in TOC
\renewcommand\appendixtocname{Anhang}

%Environments und Newcommands:
% Allgemein:
% zu zeigen symbol
\newcommand{\zz}{\fontfamily{cmss} \selectfont{Z\kern-.61em\raise-0.7ex\hbox{Z}:}}
% build over
\newcommand{\bov}[2]{\buildrel{#2} \over{#1}}
% better looking := (defined as)
\newcommand*{\defeq}{\mathrel{\vcenter{\baselineskip0.5ex \lineskiplimit0pt \hbox{\scriptsize.}\hbox{\scriptsize.}}}=}
\newcommand*{\eqdef}{=\mathrel{\vcenter{\baselineskip0.5ex \lineskiplimit0pt \hbox{\scriptsize.}\hbox{\scriptsize.}}}}

\newcommand{\hfw}{\color{RubineRed}\tx{ $\star$hier fehlt was$\star$ } \color{black}}
\def\checkmark{\tikz\fill[scale=0.4](0,.35) -- (.25,0) -- (1,.7) -- (.25,.15) -- cycle;} 

%Text:
\newcommand{\tx}[1]{\textrm{#1}}
\newcommand{\const}{\tx{const.}}

\newcommand{\ul}[1]{\underline{#1}}
\newcommand{\ol}[1]{\overline{#1}}
\newcommand{\ub}[1]{\underbrace{#1}}
\newcommand{\ob}[1]{\overbrace{#1}}

%Mathe:
\newcommand{\verteq}{\rotatebox{90}{$\,=$}}
\newcommand{\equalto}[2]{\underset{\scriptstyle\overset{\mkern4mu\verteq}{#2}}{#1}}
\newcommand{\equaltoup}[2]{\overset{\scriptstyle\underset{\mkern4mu\verteq}{#2}}{#1}}
\newcommand{\custo}[3]{\underset{\scriptstyle\overset{\mkern4mu\rotatebox{-90}{$\,#1$}}{#3}}{#2}}
\newcommand{\custoup}[3]{\overset{\scriptstyle\overset{\mkern4mu\rotatebox{-90}{$\,#1$}}{#3}}{#2}}
\newcommand{\casess}[4]{\left\{ \begin{array}{ll} {#1} & {#2} \\ {#3} & {#4} \end{array} \right.}
\newcommand{\diff}{\mathrm{d}}
\newcommand{\vphi}{\varphi}

%Spezielles:
%Theo:
\newcommand{\lag}{\mathcal{L}}
\newcommand{\ham}{\mathcal{H}}
\newcommand{\prt}[2]{\frac{\partial #1}{\partial #2}}
% new
\newcommand{\eofr}{\vec{E}(\vec{r})}
\newcommand{\pofr}{\Phi(\vec{r})}
\newcommand{\delfkt}{\delta(\vec{r} - \vec{r}_0)}
\newcommand{\gre}{\mathcal G(\vec{r},\vec{r}')}
\newcommand{\dd}{\mathrm{d}}


%LA:
\newenvironment{bew}[1]{\subsection{Bew: #1}}{\hfill$\square$}
\newcommand{\Bew}[2]{\begin{bew}{#1}#2\end{bew}}
\newcommand{\enph}{F: V \to V \textrm{ Endomorphismus}}
\newcommand{\im}{\tx{im}}
\newcommand{\spa}{\tx{span}}
\newcommand{\adj}{\tx{adj}}
\newcommand{\grad}{\tx{grad}}
\newcommand{\ord}{\tx{ord}}
\newcommand{\basis}[3]{\{#1_{#2}, \dots, #1_{#3}\}}
\newcommand{\ska}[2]{\langle #1 , #2 \rangle}
\newcommand{\dmat}[3]{\begin{pmatrix} #1_{#2}&&\\ &\ddots& \\ && #1_{#3} \end{pmatrix}}

%Ex:
\newcommand{\kq}{\frac{1}{4\pi\epsilon_0}}
\newcommand{\uind}{U_{\tx{ind}}}
\newcommand{\folie}[1]{\color{gray}[Folie: #1]\color{black}}

% ANDREZ
\newcommand{\summ}[2]{\sum_{#1}^{#2}}
\newcommand{\intt}[2]{\int_{#1}^{#2}}
\renewcommand{\vec}[1]{\bm{#1}}
\newcommand{\lcom}[1]{\color{MidnightBlue}#1\color{black}}
\renewcommand{\epsilon}{\varepsilon}
\newcommand{\vabla}{\vec{\nabla}}
\newcommand{\bei}[1]{\emph{Beispiel:}}
\newcommand{\bem}[1]{\emph{Bemerkung:}}
\renewcommand{\paragraph}[1]{\subsubsection{#1}}

% DANIIL
\newcommand{\eq}{\,=\,}

% Patrick
\newcommand{\vepsilon}{\varepsilon}

% Boxen:
\tcbuselibrary{theorems}

% mahlt eine box nur um den text mit tittel
\newtcbox{\fribox}[1]{nobeforeafter,colback=white,colframe=red!75!black,fonttitle=\bfseries,title=#1,sharp corners,tcbox raise base}

% mahlt eine große box um alles mit tittel
\newcommand{\frbox}[2]{\begin{tcolorbox}[colback=white,colframe=red!75!black,fonttitle=\bfseries,title=#1]#2\end{tcolorbox}}

% mahlt eine box nur um den text
\newtcbox{\ribox}{nobeforeafter,colback=white,colframe=red!75!black,sharp corners,tcbox raise base}

% mahlt eine große box um alles was drinnen ist
\newcommand{\rbox}[1]{\begin{tcolorbox}[colback=white,colframe=red!75!black]#1\end{tcolorbox}}

% mahlt eine box um mathe innerhalb mathmode
\newcommand{\rmbox}[1]{\tcboxmath[colback=white,colframe=red!75!black]{#1}}

% super box (looks like regular boxed but wraps around anything)
\newenvironment{supbox}{\begin{tcolorbox}[colback=white,colframe=black,sharp corners,boxrule=.5pt]}{\end{tcolorbox}}

\newcommand{\bbb}[2]{\begin{tcolorbox}[colback=white,colframe=black,fonttitle=\bfseries,title=#1,sharp corners,tcbox raise base]#2\end{tcolorbox}}

% array type box with title
\newenvironment{zebox}[1]{\begin{array}{|c|}
		\multicolumn{1}{l}{\tx{#1}} \\
		\hline
		\displaystyle
	}{\\ \hline
\end{array}}

% evtl: \renewcommand{\boxed}{\rmbox}
%\renewcommand{\thechapter}{\Roman{chapter}}
\hbadness=99999








\begin{document}
\title{
	{\Huge Theoretische Physik II\\[3pt]Elektrodynamik}\\[1em]
	{\Large Vorlesung von Prof. Dr. Michael Thoss im Wintersemester 2018}}
\author{Daniil Aktanka \\ Patrick Munnich \\ Andréz Gockel}

\maketitle
\tableofcontents



\chapter{Grundbegriffe}

\paragraph{Ladung}
\begin{align*}\hspace{1cm}
\llap{\text{diskrete Ladungsverteilung}} \qquad Q &= \sum_{i = 1}^{n} q_i \\
\llap{\text{kontinuierliche Ladungsverteilung}} \qquad Q &= \int_{V} \rho ( \vec{r} ) \; \text{d}^3r \\
\llap{\text{Punktladung}} \quad\ \rho ( \vec{r} ) &= q \delta ( \vec{r} - \vec{r}_0)\\
\llap{$n$ Punktladungen}\quad \rho(\vec{r}')\, &=\, \sum_{j=1}^{n}\, q_j\, \delta (\vec{r}'-\,\vec{r}_j)
\end{align*}

\paragraph{Coulomb’sches Gesetz}
\begin{align*}
\llap{\text{zwei Punktladungen}} \qquad \vec{F}_{12} &= \frac{1}{4 \pi \epsilon_0} \, q_1 q_2 \, \frac{\vec{r}_1 - \vec{r}_2}{\left| \vec{r}_1 - \vec{r}_2\right|^3} = -\vec{F}_{21} \\
\llap{$n$ Punktladungen} \qquad \vec{F}_1 &= \frac{1}{4 \pi \epsilon_0} \, q_1 \sum_{j=2}^{n} q_j \, \frac{\vec{r} - \vec{r}_j}{\left| \vec{r} - \vec{r}_j\right|^3} \\
\llap{Beziehung zur $E$-Feld} \quad\, \vec{F}(\vec{r}) \, &=\, q \, \vec{E}(\vec{r})\\
\llap{Beziehung zur Potential} \quad \vec{F}(\vec{r}) &=\, -\,\vec{\nabla}\, q\,\varphi(\vec{r})
\end{align*}

\paragraph{Elektrisches Feld}
\begin{align*}
\llap{\text{im bel. Raumpunkt}} \qquad \vec{E} (\vec{r}) &= \frac{q}{4 \pi \epsilon_0} \, \frac{\vec{r} - \vec{r}_0}{\left| \vec{r} - \vec{r}_0\right|^3} \\
\llap{\text{diskret}} \qquad \vec{E}(\vec{r}) &= \frac{1}{4 \pi \epsilon_0} \, \sum_{j=2}^{n} q_j \, \frac{\vec{r} - \vec{r}_j}{\left| \vec{r} - \vec{r}_j\right|^3} \\
\llap{\text{kontinuierlich}} \qquad \vec{E}(\vec{r}) &= \frac{1}{4 \pi \epsilon_0} \int \rho (\vec{r}') \, \frac{\vec{r} - \vec{r}'}{\left| \vec{r} - \vec{r}'\right|^3} \; \text{d}^3r' \\
\llap{\scriptsize(Identität)} \qquad\ \ \Bigg\downarrow \quad \rlap{$ \frac{\vec{r} - \vec{r}'}{\left| \vec{r} - \vec{r}'\right|^3}  \, = \, -\nabla_r\, \frac{1}{\left| \vec{r} - \vec{r}'\right|} \quad \text{\scriptsize{wobei}}\quad \vabla_r\frac{1}{\left|\vec{r}-\vec{r}'\right|}=-\vabla_{r'}\frac{1}{\left|\vec{r}-\vec{r}'\right|} $ } \\
\llap{\text{Beziehung zur Potential}} \qquad \vec{E} (\vec{r}) &= \; - \nabla \, \varphi (\vec{r})\\
\implies \qquad \vec{\nabla}\times& q\vec{E}\, =\, 0 \qquad \rlap{\scriptsize{d.h., die Coulomb-Kraft ist konservativ}}
\end{align*}

\paragraph{Skalare Elektrische Potential}
\begin{align*}\hspace{4cm}
\llap{im bel. Raumpunkt kontinuierlich} \qquad \varphi(\vec{r}) \, &= \, \frac{1}{4\pi\epsilon_0} \, \int \frac{\rho(\vec{r}')}{\left| \vec{r} - \vec{r}'\right|}\;\mathrm{d}^3r'\\
\llap{im bel. Raumpunkt diskret} \qquad \varphi(\vec{r}) \, &= \, \frac{1}{4\pi\epsilon_0} \, \sum_{j=1}^{n}\, \frac{q_j}{\left| \vec{r} - \vec{r}_j\right|}\\
\llap{Spannung / Potentialdifferenz} \qquad U(\vec{r},\, \vec{r}_0)\, &=\, \varphi(\vec{r})\, -\, \varphi(\vec{r}_0)\, =\, -\, \int_{\vec{r}_0}^{\vec{r}}\, \vec{E}(\vec{r}')\; \mathrm{d}\vec{r}'\\
\end{align*}

\paragraph{Operator Nomenklatur}
\begin{align*}
\text{div}\, \vec{A} \, &= \, \vec{\nabla}\cdot \vec{A} \qquad \rlap{\scriptsize{$\begin{cases} 
		>0 \, , & \mathrm{Quelle} \\
		=0 \, , & \mathrm{quellenfrei}\\
		<0\, , & \mathrm{Senke}
	\end{cases}$}} \\
\text{grad}\,\vec{A} \, &= \, \vec{\nabla}\vec{A} \\
\llap{\scriptsize{($eng.$ curl)}} \quad \text{rot}\, \vec{A} \, &= \, \vec{\nabla}\times \vec{A} \qquad \rlap{\scriptsize{Wirbel}}
\end{align*}

\paragraph{Produktformeln}
$f, g$ sind skalare Felder, $\vec{F}, \vec{G}$ sind Vektor Felder:

\begin{align*}
 	\vabla(fg) &= f \vabla (g) + g \vabla (f)\\
 	\vabla \cdot (f \vec{G}) &= f \vabla \cdot (\vec{G}) + \vec{G} \cdot \vabla (f)\\
 	\vabla (\vec{F} \times \vec{G}) &= \vec{G} \cdot \vabla \times (\vec{F}) - \vec{F} \cdot \vabla \times (\vec{G})\\
 	\vabla \times (f \vec{G}) &= f \vabla \times (\vec{G}) - \vec{G} \times \vabla
\end{align*}

\paragraph{Identitäten}

\begin{align*}
	\vec{\nabla}\times(\vec{\nabla}\vec{F})&=\vec{\nabla}(\vec{\nabla}\cdot\vec{F})-\Delta\vec{F}\\
	\vec{\nabla}\times(\vec{\nabla}f)&=\vec{0}\\
	\vec{\nabla}\cdot(\vec{\nabla}\times\vec{F})&=0\\
	\vec{\nabla}\times(\vec{a}\times\vec{\nabla}f)&=\vec{a}\Delta f-\vec{\nabla}(\vec{a}\cdot\vec{\nabla}f)
\end{align*}



\paragraph{Additionstheorem}
\[\frac{1}{|\vec{r}-\vec{r}'|}\,=\,\frac{1}{\sqrt{r^2+r'^2-2r\,r'\,\cos\theta}}\,=\,\sum_{l=0}^{\infty}\frac{r_<^l}{r_>^{l+1}}P_l(\cos\theta)\qquad\rlap{\scriptsize{in Kugelkoordinaten}}\]
\[r_<\eq \mathrm{min}(r,r')\ ,\qquad r_>\eq \mathrm{max}(r,r')\]


\paragraph{Kugelkoordinaten}
$$
\begin{pmatrix}
	x \\ y\\ z
\end{pmatrix}
=
\begin{pmatrix}
	r \sin \theta \cos \varphi\\
	r \sin \theta \sin \varphi\\
	r \cos \theta
\end{pmatrix}
, \qquad \int\limits_{-\infty}^{+\infty} f(\vec{r})\ \textrm{d}r = \int\limits_{0}^{2 \pi} \int\limits_{0}^{\pi} \int\limits_{0}^{\infty} f(r,\theta,\varphi) r^2 \sin \theta\ \textrm{d}r \textrm{d}\theta \textrm{d}\varphi
$$

wobei:\\
$r\in \lbrack0\, ,\,\infty)\, ,\quad \Theta\in\lbrack 0\, ,\, \pi\rbrack\, , \quad \varphi\in\lbrack 0\, ,\, 2\pi) $


\chapter{Elektrostatik}

\section{Einführung}
\paragraph{Dirac'sche Delta-Funktion}
\subparagraph{Definition}

\begin{align*}
\int_{V} \delta(\vec{r}-\vec{r}_0) \; \text{d}^3r &\coloneqq
	\begin{cases} 
		1 \, , & r_0 \in V \\
		0 \, , & \text{sonst}
	\end{cases} \\
\delta(\vec{r}-\vec{r}_0) &= 0 \quad \ \forall \: \vec{r} \neq \vec{r_0}
\end{align*}

\emph{Bemerkung:} Die $\delta$-Funktion ist keine Funktion im üblichen mathematischen Sinne. Man bezeichnet sie deshalb als \textbf{uneigentliche Funktion} oder als \textbf{Distribution}. Heuristisch:

\begin{align*}
&\delta(x) = 
	\begin{cases}
		+ \infty \, , & x = 0 \\
		0 \, , & x \neq 0
	\end{cases} \\
&\int_{-\infty}^{\infty} \delta (x) \, dx = 1
\end{align*}

\subparagraph{Formeln}
\begin{align*}
\int_{a}^{b} f(x) \, \delta (x-x_0) \, dx &= 
	\begin{cases}
		f(x_0) \, , & a < x_0 < b \\
		\frac{1}{2} f(x_0) \, , & x_0 = a \lor b \\
		0 \, & \text{sonst}
	\end{cases} \\
`` \ f(x) \, \delta '(x-x_0) &= -f'(x) \, \delta (x-x_0) \ " \qquad \rlap{\text{\scriptsize{(heuristisch)}}} \\
\delta (x - x_0) &= \frac{d}{dx} \Theta (x - x_0) \qquad \rlap{\scriptsize{($ \Theta $ \, \text{sei die Stufenfunktion)}}}\\
\delta(\vec{r}-\vec{r}')\,&=\,-\,\frac{1}{4\pi}\Delta\,\frac{1}{|\vec{r}-\vec{r}'|}
\end{align*} 

\subparagraph{Mehrdimensionale Delta-Funktion}
\begin{align*}\hspace{1.5cm}
\llap{\text{Kartesisch (x,y,z)}} \qquad \delta(\vec{r} - \vec{r}_0) &= \delta(x - x_0) \, \delta(y - y_0) \, \delta(z - z_0) \\
\llap{\text{Kugel $ (r, \theta, \varphi) $}} \qquad \delta(\vec{r} - \vec{r}_0) &= \frac{1}{r^2_0 \sin\theta_0} \, \delta(r - r_0) \, \delta(\theta - \theta _0) \, \delta(\varphi - \varphi _0) \\
\llap{\text{Zylinder $ (\rho , \phi , z) $}} \qquad \delta(\vec{r} - \vec{r}_0) &= \frac{1}{\rho _0} \, \delta(\rho - \rho _0) \, \delta(\varphi - \varphi _0) \, \delta(z - z _0)
\end{align*}

\section{Integralsätze}
\paragraph{Einleitung: Fluss}
\subparagraph{Definition}
Sei $\vec{a}(\vec{r})\,=\,(a_1(\vec{r}),\,a_2(\vec{r}),\, a_3(\vec{r}))$ ein Vektorfeld, $V$ ein Volumen und $S(V)$ die Oberfläche. Dann heisst $\Phi_S(\vec{a})$ der Fluss ($eng.$ flux) von $\vec{a}(\vec{r})$ durch die Fläche S wenn es gilt:

\[\Phi_S(\vec{a})\,=\,\int_{S}\,\vec{a}(\vec{r})\;\mathrm{d}\vec{f}\]

\subparagraph{Geschlossene Fläche} Das Oberflächenintegral über eine geschlossene Fläche wird durch ein spezielles Integralzeichen symbolisiert:
\[\Phi_S(\vec{a})\,=\,\oint_S\,\vec{a}(\vec{r})\;\mathrm{d}\vec{f}\]

\paragraph{Satz von Stokes}
\subparagraph{Bedeutung}
Ganz allgemein gesagt handelt es sich um einen sehr grundlegenden Satz über die Integration von Differentialformen. Es geht darum, $n$-dimensionale Volumenintegrale über das Innere in $(n-1)$-dimensionale Randintegrale über die Oberfläche des Volumenstücks umzuwandeln. Für uns sind die Spezialfälle am wichtigsten, bei denen der Gauß'sche Satz und der Kelvin-Stokes'sche Satz (Rotationssatz).

\subparagraph{Rotationssatz} (\textit{auch} Klassische Integralsatz von Stokes) \hfill \break

Seien $\vec{a}(\vec{r})$ ein hinreichend oft differenzierbares Vektorfeld und F eine Fläche mit dem Rand $C(F) = \delta F$. Dann gilt:

$$\int_{F} \vabla \times \vec{a}\; \mathrm{d}\vec{f} = \int_{\partial F}\vec{a}\; \mathrm{d}\vec{r}$$

\paragraph{Gauß'sche Satz}
($eng.$ Divergence Theorem)

\subparagraph{Definition} Der elektrische Nettofluss $\Phi$ durch eine hypothetische geschlossene Oberfläche $S$ ist gleich $\frac{1}{\epsilon_0}$ mal die elektrische Nettoladung $Q$ innerhalb dieser geschlossenen Oberfläche.

\subparagraph{Integrale Form}
\[\Phi\,=\,\oint_S\, \vec{E}\; \mathrm{d}\vec{f}\,=\,\frac{Q}{\epsilon_0}\]
Wobei $Q$ die Gesamtladung innerhalb $V$ ist.

\subparagraph{Differentielle Form}
\[\vec{\nabla}\cdot\vec{E}\,=\,\frac{\rho}{\epsilon_0}\]
Wobei $\rho$ die Gesamtladungsdichte (pro Einheit Volumen) ist.

\subparagraph{Beziehung}
$$\int_{V} \vabla \cdot \vec{E}\;\mathrm{d}^{3}r = \oint_{S}\vec{E}\;\mathrm{d}\vec{f}$$
\hfill \break

\paragraph{Greensche Sätze}
Die Green’sche Sätze (Identitäten) lassen sich aus Anwendungen des Gauß’schen Satzes ableiten. Hierfür wurde die folgende Definition der Normalableitung benutzt.

\subparagraph{Normalableitung} \hfill \break
Sei $\psi$ ein mindestens zweimal stetig differenzierbares, skalares Feld auf einer Fläche $S(V)$. Sei $\vec{n}(\vec{r})$ eine (ortsabhängige) Flächennormale. Dann gilt:
\[\vec{\nabla}\psi\cdot\vec{n}\,\equiv\,\frac{\partial\psi}{\partial n}\]

\subparagraph{1. Green'sche Identität}
\[\int_V(\varphi\Delta\psi\,+\,(\vabla\psi\cdot\vabla\varphi))\;\mathrm{d}^3r\,=\,\oint_{S}\varphi\frac{\partial\psi}{\partial n}\;\mathrm{d}f\]

Wobei $\varphi$ und $\psi$ Skalarfunktionen sind, angenommen sei $\varphi$ einfach und $\psi$ zweifach stetig differenzierbar.

\subparagraph{2. Green'sche Identität}
\[\int_V (\varphi\Delta\psi-\psi\Delta\varphi)\;\mathrm{d}^3r\,=\,\oint_{S}\bigg(\varphi\frac{\partial\psi}{\partial n}-\psi\frac{\partial\varphi}{\partial n}\bigg)\;\mathrm{d}^3f\]

Wobei $\varphi$ und $\psi$ beide zweifach stetig differenzierbar sind.

\bigskip
\emph{Bemerkung:} Die beiden oben genannten Green'schen Identitäten sind Spezialfällen und gelten nur unter Bedienungen.

\section{Helmholtz-Zerlegung}
\paragraph{Bedeutung}
Zusammengefasst besagen der Satz, dass unter gewissen Voraussetzungen jedes Vektorfeld $\vec{a}(\vec{r})$ eindeutig durch sein Quellenfeld $\vabla\cdot\vec{a}(\vec{r})$ und sein Wirbelfeld $\vabla\times\vec{a}(\vec{r})$ bestimmt ist. Oder anders ausgedrückt: (fast) jedes Vektorfeld lässt sich eindeutig als Summe eines wirbelfreien und eines quellenfreien Anteils darstellen.

\paragraph{Aussage des Theorems}
Sei $F$ ein zweifach stetig differenzierbares Vektorfeld über ein beschränktes Gebiet im $\mathbb{R}^3$ und $S$ eine Oberfläche welche dieses Gebiet umschließt. Dann kann $F$ in eine curl-freie Komponente und eine divergenzfreie Komponente zerlegt werden:

\[\vec{F}\,=\,-\vabla\Phi+\vabla\times\vec{A}\]

Wobei die beiden einander ergänzenden Potentiale $\Phi$ und $A$ lassen sich durch die folgenden Integrale aus dem Feld $F$, über das die Felder enthaltende Volumen $V$, gewinnen:

\[
\vec{\Phi}(\vec{r})\,=\,\frac{1}{4\pi}\int_V \frac{\vabla '\cdot \vec{F}(\vec{r}')}{\left|\vec{r}-\vec{r}'\right|}\;\mathrm{d}V' \qquad\quad
\vec{A}(\vec{r})\,=\,\frac{1}{4\pi}\int_V \frac{\vabla '\times \vec{F}(\vec{r}')}{\left|\vec{r}-\vec{r}'\right|}\;\mathrm{d}V'
\]

\subparagraph{Schlussfolgerungen}
\begin{itemize}
	\item Ein wirbelfreies Feld ($\vabla\times\vec{a}\,=\,0$) ist ein Gradientenfeld.
	\item Ein quellenfreies Feld ($\vabla\cdot\vec{a}\,=\,0$) ist ein Rotationsfeld.
\end{itemize}

\newpage
\section{Maxwell-Gleichungen der Elektrostatik}
\paragraph{Differenzielle Darstellung}
\begin{align*}
\llap{physikalischer Gauß’scher Satz}\qquad\vabla\cdot\vec{E}\,&=\,\frac{1}{\epsilon_0}\rho\\
\llap{Faraday'sche Induktionsgesetz}\qquad\vabla\times\vec{E}\,&=\,0
\end{align*}

\paragraph{Integrale Darstellung}
\begin{align*}
\llap{physikalischer Gauß’scher Satz}\qquad\int_S \vec{E}\cdot\mathrm{d}\vec{f}\,&=\,\frac{1}{\epsilon_0}\,q(V)\\
\llap{Faraday'sche Induktionsgesetz}\qquad\int_{\partial F} \vec{E}\cdot\mathrm{d}\vec{r}\,&=\,0
\end{align*}

\paragraph{Poisson Gleichung (Einführung)}
Durch die Einführung  des skalaren Potentials $\phi(\vec{r})$ lassen sich die beiden Maxwell-Gleichungen zusammenfassen zur so genannten Poisson-Gleichung.

\[\Delta\varphi(\vec{r})\,=\,\frac{1}{\epsilon_0}\,\rho(\vec{r})\]

Die Lösung dieser linearen, inhomogenen, partiellen Differenzialgleichung 2. Ordnung bezeichnet man als das Grundproblem der Elektrostatik. Ist der Raumbereich ladungsfrei, dann bekommt man die Laplace-Gleichung:

\[\Delta\varphi(\vec{r})\,=\,0\]

Die allgemeine Lösung der Poisson-Gleichung lässt sich als Summe einer speziellen Lösung der Poisson-Gleichung und der allgemeinen Lösung der Laplace-Gleichung darstellen.

\section{Elektrostatische Feldenergie}
\paragraph{Definition}
Die Energie einer auf einen endlichen Raumbereich beschränkten Ladungskonfiguration $\rho(\vec{r})$ entspricht der Arbeit, die notwendig ist, um Ladungen aus dem Unendlichen $(\varphi(\infty)\,=\,0)$ zu dieser Konfiguration zusammenzuziehen.

\paragraph{Formeln}
\begin{align*}
\llap{diskret}\qquad W\,&=\,\frac{1}{4\pi\epsilon_0}\sum_{i=2}^{N}\sum_{j=1}^{i-1}\frac{q_i q_j}{\left|\vec{r}_i - \vec{r}_j \right|}\,=\,\frac{1}{8\pi\epsilon_0}\sum_{\substack{i,j=1\\ i\neq j}}^{N}\,\frac{q_i q_j}{\left|\vec{r}_i - \vec{r}_j \right|}\\
\llap{kontinuierlich}\qquad W\,&=\,\frac{1}{8\pi\epsilon_0}\int\int\frac{\rho(\vec{r})\rho(\vec{r}')}{\left|\vec{r}_i - \vec{r}_j \right|}\;\mathrm{d}^3r\;\mathrm{d}^3r'\,=\,\frac{1}{2}\int\rho(\vec{r})\varphi(\vec{r})\;\mathrm{d}^3r\\
W\,&=\,\frac{\epsilon_0}{2}\int\left|\vec{E}(\vec{r})\right|^2\;\mathrm{d}^3r\\
\llap{Energiedichte}\qquad w\,&=\,\frac{\epsilon_0}{2}\left|\vec{E}(\vec{r})\right|^3
\end{align*}

\newpage
\section{Einfache elektrostatische Probleme}
\paragraph{Elektrisches Potential einer homogen geladenen Kugel}
Zur Lösung des Problems können wir das Poisson-Integral verwenden. \cite{homkugelpot}

\paragraph{Methode der Spiegelladungen}
($eng.$ Method of Image Charges)

\subparagraph{Beschreibung}
Die Spiegelladung (oder Bildladung) ist eine gedankliche Hilfsstütze, um das Verhalten einer Ladung $Q$ vor einem leitenden Körper im Abstand $R$ zu veranschaulichen. Beim Fall eines Leiters wird die gesamte influenzierte Ladung dafür anschaulich zu einer Punktladung zusammengefasst. Aus Symmetriegründen wird diese Punktladung als Spiegelladung bezeichnet. (für physikalische Grundlagen siehe Anhang\ref{imagechargestheory})

\subparagraph{Punktladung und die ungeladene Metallkugel}
Sein $r$ und $r'$ die Abstände zur realen- und Spiegelladung, mit Koordinatenursprung im Zentrum des Kugels mit Radius $R$. Dann liegen die beide Ladungen auf einer Gerade, wobei (mittels Kreisspiegelung) gilt:

\begin{align*}
r'\,&=\,\frac{R^2}{r}\\
q'\,&=\,q\,\frac{R}{r}
\end{align*}

\emph{Bem.} Die Platzierung elektrischer Ladungen sollte irrelevant sein.

\section{Randwertprobleme der Elektrostatik}
\paragraph{Formulierung des Problems}
Das Ziel ist Lösung der Poisson-Gleichung zu finden, was man als das Grundproblem der Elektrostatik bezeichnet. Falls die das Potential $\varphi(\vec{r})$ erzeugende Ladungsdichte $\rho(\vec{r}')$ bekannt ist und keine spezielle Randbedinungen auf Grenzflächen im Endlichen zu erfüllen sind, dann reicht die folgende allgemeine Lösung völlig aus:

\[\varphi(\vec{r})\,=\,\frac{1}{4\pi\epsilon_0}\int\frac{\rho(\vec{r}')}{\left|\vec{r}-\vec{r}'\right|}\;\mathrm{d}^3r'\qquad\rlap{Poisson-Integral}\]

Ist $\rho$ räumlich begrenzt, so gilt insbesondere

\[\varphi\xrightarrow[r\rightarrow\infty]{}0\ ;\qquad\vabla\varphi\xrightarrow[r\rightarrow\infty]{}0\]
\noindent
Dies ist jedoch bei vielen praktischen Problemen nicht der eigentliche Ausgangspunkt.

\paragraph{Definition}
\begin{itemize}[leftmargin=3cm]
	\item[\textbf{Gegeben:}] $\rho(\vec{r}')$ in einem gewissen Raumbereich $V$, $\varphi$ oder $\frac{\partial\varphi}{\partial n}\,=\,-\vec{E}\cdot \vec{n}$
	\item[\textbf{Gesucht:}] Das skalare Potential $\varphi(\vec{r})$ in allen Punkten $\vec{r}$ des interessierenden Raumbereichs $V$
\end{itemize}

\paragraph{Klassifikation der Randbedingungen}
\begin{align*}\hspace{2cm}
\llap{Dirichlet-Randbedingungen}\qquad&\varphi\ \text{auf}\ S(V)\ \text{gegeben}\\
\llap{Neumann-Randbedingungen}\qquad&\frac{\partial\varphi}{\partial n}\,=\,-\vec{n}\cdot\vec{E}\ \text{auf}\ S(V)\ \text{gegeben}
\end{align*}
\noindent
Von gemischten Randbedingungen spricht man, wenn diese auf S(V) stückweise Dirichlet- und stückweise Neumann-Charakter haben.

\paragraph{Green'sche Funktion}
\subparagraph{Mathematische Definition}
Die Green'sche Funktion $G(x, x_0)$ eines linearen Differentialoperators $L\,=\,L(x)$, der auf Verteilungen über einer Teilmenge des euklidischen Raums an einem Punkt $x_0$ wirkt, ist eine Lösung von:

\[LG(x,x_0)\,=\,\delta(x-x_0)\]

wobei $\delta$ die Deltafunktion ist.

\subparagraph{Lösung der Poisson-Gleichung für eine Punktladung $q = 1$} Es gilt:
\[\qquad \Delta_r G(\vec{r},\vec{r}')=-\frac{1}{\epsilon_0}\delta(\vec{r}-\vec{r}')\]

Wobei die Funktion in $\vec{r}$ und $\vec{r}'$ symmetrisch ist; d.h., wir können den Laplace-Operator auf die Variable $\vec{r}'$ wirken lassen. Die $\delta$-Umformung liefert:
\[G(\vec{r},\vec{r}')\,=\,\frac{1}{4\pi\epsilon_0}\frac{1}{\left|\vec{r}-\vec{r}'\right|}+f(\vec{r},\vec{r}')\ ,\qquad \nabla_r f(\vec{r},\vec{r}')\,=\,0\qquad\rlap{Bedingung}\]

Für $\vec{r}\in V$ allgemeine Lösung gilt:
\[\varphi(\vec{r})\,=\,\int_V\rho(\vec{r}')G(\vec{r},\vec{r}')\;\dd^3r'-\epsilon_0\int_{S(V)}\Bigg(\underbrace{\varphi(\vec{r}')\frac{\partial G}{\partial n'}}_{\mathclap{\substack{{=\,0}\\{\text{Neumann}}}}}-\underbrace{G(\vec{r},\vec{r}')\frac{\partial\varphi}{\partial n'}}_{\mathclap{\substack{{=\,0}\\{\text{Dirichlet}}}}}\Bigg)\;\dd f'\]

Wobei je nach Bindungstyp wird ein Teil des Integrals zu null.

\subparagraph{Green'sche Funktion für die Kugel}
\begin{align*}
G_\mathrm{D}(\vec{r},\vec{r}')\,&=\,\frac{1}{4\pi\epsilon_0}\Bigg(\frac{1}{\left|\vec{r}-\vec{r}'\right|}-\frac{1}{\left|\frac{r'}{R}\vec{r}-\frac{R}{r'}\vec{r}'\right|}\Bigg)\eq \frac{1}{q}\Phi(\vec{r})\qquad\rlap{\scriptsize{mit $\Phi$ Potential hom. gel. Kugel}}\\
&=\,\frac{1}{4\pi\epsilon_0}\Bigg[ (r^2+r'^2-2r\,r'\vec{e}_r\,\cdot\vec{e}_{r'})^{-\nicefrac{1}{2}}+\Big( \frac{r^2r'^2}{R^2}+R^2-2r\,r'\,\vec{e}_r\cdot\vec{e}_{r'} \Big)^{-\nicefrac{1}{2}} \Bigg]
\end{align*}

\paragraph{Separation der Variablen}
Wir suchen nach weiteren Lösungsmethoden für die Poisson-Gleichung,
\[\Delta\varphi(\vec{r})\,=\,-\frac{1}{\epsilon_0}\rho(\vec{r})\]
die eine lineare, partielle, inhomogene Differenzialgleichung zweiter Ordnung für ein Gebiet darstellt, auf dessen Rand gewisse Bedingungen vorgeschrieben sind. Benutze:
\hfill\break
\[\llap{Separationsansatz}\qquad\varphi(\vec{r})\,=\varphi(x,y,z)\,=\,f(x)\,g(y)\,h(z)\]

\pagebreak
\paragraph{Lösung der Laplace-Gleichung in Kugelkoordinaten}
\subparagraph{Exkurs: Funktionen der Kugelfläche} Siehe Anhang \ref{sphericalfunctions} 
\subparagraph{Methode} Wir suchen eine Lösung der Laplace-Gleichung $\Phi(r,\theta,\varphi)$ in Kugelkoordinaten. Wir benutzen den Entwicklungssatz, um das Potential $\Phi$ durch diese Funktionen auszudrücken:
\[\llap{gesucht ist die Lösung von:}\qquad\Delta\Phi(r,\theta,\varphi)\,=\,0\]
\[\llap{benutze:}\qquad\Phi(r,\theta,\varphi)\,=\,\sum_{l=0}^{\infty}\sum_{m=-l}^{+l}R_{lm}(r)Y_{lm}(\theta,\varphi)\]
Wir wenden darauf den Laplace-Operator an. Dies führt zu der so genannten Radialgleichung:
\[\frac{1}{r^2}\frac{\dd R}{\dd r}\left[r^2\frac{\dd R}{\dd r}\right]-\frac{l(l+1)}{r^2}R\,=\,0\]
Mit einem Ansatz gelöst hat das Potential $\Phi$ damit die allgemeine Gestalt:
\begin{align*}\hspace{3cm}
\llap{allgemeine Gestalt}\qquad\Delta\Phi(r,\theta,\varphi)\,&=\,\sum_{l=0}^{\infty}\sum_{m=-l}^{+l}(A_{lm}r^l+B_{lm}r^{-(l+1)})Y_{lm}(\theta,\varphi)\\
\llap{azimutaler Symmetrie}\qquad\Delta\Phi(r,\theta)\,&=\,\sum_{l=0}^{\infty}(2l+1)\left[A_{lm}r^l+B_{lm}r^{-(l+1)}\right]P_l(\cos\theta)
\end{align*}

\paragraph{Potential einer Punktladung}
Es gibt eine alternative Multipolentwicklung, wenn man den $\nicefrac{1}{|\vec{r}-\vec{r}'|}$ Term nach Kugelflächenfunktionen entwickelt. Wir denken uns eine Kugel mit dem Koordinatenursprung als Mittelpunkt und dem Radius $r_0$.  Es gilt für das Potential der Punktladung:
\[\Phi(\vec{r})\,=\,\frac{q}{4\pi\epsilon_0 |\vec{r}-\vec{r}_0|}\eq\frac{q}{\epsilon_0 r_>}\sum_{l=0}^{\infty}\sum_{m=-l}^{+l}\frac{l}{2l+1}\left(\frac{r_<}{r_>}^l\right) Y_{lm}^*(\theta_0,\varphi_0)Y_{lm}(\theta,\varphi)\]
wobei die folgende Notation eingeführt wurde:
\begin{gather*}
\text{innerhalb}\ (r<r_0):\qquad r\eq r_<\ , \quad r_0\eq r_>\\
\text{außerhalb}\ (r>r_0):\qquad r\eq r_>\ , \quad r_0\eq r_<
\end{gather*}

\pagebreak
\subsection{Multipolentwicklung}
\sloppy
Eine Multipolentwicklung ist eine mathematische Reihe einer winkelabhängigen Funktion. In Physik ist es normalerweise die Reihenentwicklung eines Potentials, bei der verschiedene Multipol-Momente auftreten. Man unterscheidet zwischen kartesischer und sphärischer Multipolentwicklung.

\paragraph{Definition}
\subparagraph{Multipol}- eine beliebige räumliche Anordnung von (elektrischen oder anderen) diskreten Ladungen oder, im magnetischen Fall, Dipolen. Durch eine Multipolentwicklung kann ein Potentialfeld durch Punktladungen angenähert werden.

\paragraph{Methode}
Wir betrachten eine statische, lokalisierte Ladungsverteilung:
\[\rho(\vec{r})\eq \rho(r,\theta,\varphi)\eq \begin{cases}
	\mathrm{beliebig} \, , & r<R_0 \\
	0 \, , & r>R_0 \\
\end{cases}\]

Im Bereich $r > R_0$ kann das elektrostatische Potenzial $\Phi$ nach Potenzen von $\nicefrac{R_0}{r}$ entwickelt werden. Diese Multipolentwicklung wird im Folgenden auf zwei Arten abgeleitet.

\paragraph{Sphärische Darstellung}
Unter Verwendung von Kugelflächenfunktionen gilt für den Potential:

\begin{align*}
\Phi(\vec{r})&\eq \sum_{l=0}^{\infty}\sum_{m=-l}^{+l}\sqrt{\frac{4\pi}{2l+1}}\frac{q_{lm}}{r^{l+1}}Y_{lm}(\theta,\varphi)\\
&\eq \frac{q_{00}}{r}+\frac{q_{10}}{r^2}\cos\theta\mp\frac{1}{\sqrt{2}}\frac{q_{1,\pm 1}}{r^2}	\sin\theta\exp(\pm \mathrm{i}\varphi)\pm\dots
\end{align*}
\noindent
sphärische Multipolmomente:
\begin{align*}
q_{lm}&\eq \sqrt{\frac{4\pi}{2l+1}}\int\dd^3 r'\ \rho(\vec{r}')r'^l\,Y_{lm}^*(\theta ', \varphi ')\\
\llap{\scriptsize{zylindersymmetrische Fall}}\qquad q_{l0}&\eq\int\dd^3r\ \rho(r,\theta)r^l P_l(\cos\theta)\qquad\rlap{wobei $\rho\eq (r,\theta)\, ,\quad q_{l0}\eq \delta_{m0}$}
\end{align*}

\paragraph{Kartesische Darstellung}
Sei $\nicefrac{1}{|\vec{r}-\vec{r}'|}\eq [(x_1-x_1')^2+(x_2-x_2')^2+(x_3-x_3')^2]^{\nicefrac{1}{2}}$. Dann gilt:

\[\Phi(\vec{r})\eq \frac{q}{r}+\frac{\vec{p}\cdot\vec{r}}{r^3}+\frac{1}{2}\sum_{i,j\eq 1}^{3}Q_{ij}\frac{x_ix_j}{r^5}+\dots\qquad\tag{$r>R_0$}\]
\noindent
kartesische Multipolmomente:
\begin{align*}
q&\eq \int\dd^3r'\ \rho(\vec{r}')&&\text{Ladung}\\
p_i&\eq \int\dd^3r'\ x_i'\,\rho(\vec{r}')&&\text{Dipolmoment}\\
Q_{ij}&\eq \int\dd^3 r'\ (3x_i'x_j'-r'^2\, \delta_{ij})\rho(\vec{r}')&&\text{Quadrupolmoment}
\end{align*}

Unter orthogonalen Transformationen ist $q$ ein Skalar, $p_i$ ein Vektor und $Q_{ij}$ ein Tensor zweiter Stufe.





\newpage
\section{Elektrostatik in Materie - Dielektrika}
Siehe Skript von Markus Österle und Andréz Gockel.

%\paragraph{Not very usefull}
%\subparagraph{Gradienten eines skalaren Feldes}
%$$\vabla g(\vec{r}) = \sum_i \vec{e}_i \frac{1}{\left|\prt{\vec{r}}{u_i}\right|}\prt{}{u_i}g(\vec{r})$$
%\int_a^bu(x)v'(x)\mathrm{d}x&=[u(x)v(x)]_a^b-\int_a^bu'(x)v(x)\mathrm{d}x\\
%H(x)&\coloneqq\int_{-\infty}^x\delta(s)\mathrm{d}s=\begin{cases}0&\textrm{for }x<0\\1&\textrm{for }x\geq0

\chapter{Elektrodynamik}

\section{Maxwell-Gleichungen}
\begin{equation}
\vabla\cdot\vec{E}=\frac{\rho}{\epsilon_0}\label{Max:1}
\end{equation}

\begin{equation}
\vabla\cdot\vec{B}=0\label{Max:2}
\end{equation}

\begin{equation}
\vabla\times\vec{E}=-\frac{\partial\vec{B}}{\partial t}\label{Max:3}
\end{equation}

\begin{equation}
\vabla\times\vec{B}=\vec{J}+\mu_0\epsilon_0\frac{\partial\vec{E}}{\partial t}\label{Max:4}
\end{equation}








\renewcommand\appendixpagename{Anhang}
\begin{appendices}
\section{Methode der Spiegelladungen}\label{imagechargestheory}
Wird ein leitender Körper in ein äußeres elektrisches Feld gebracht, so stehen meistens zu Beginn die Feldlinien noch nicht senkrecht auf der Oberfläche. Dies führt zu Potentialunterschieden entlang der Oberfläche, welche die frei beweglichen Elektronen dazu bringen sich so zu verschieben, dass die Feldlinien senkrecht auf die Oberfläche treffen (äußere elektrische Felder nehmen innerhalb leitender Körper exponentiell mit der Zeit ab). Da das äußere elektrische Feld auch Potentialunterschiede im Körper verursacht, bewegen sich die Elektronen innerhalb des Körpers so, dass dort überall das gleiche Potential herrscht. Theoretisch führt das demzufolge zu Oberflächenladungsdichten. Mikroskopisch nah betrachtet halten die Elektronen zueinander einen Abstand und deshalb befinden sich die Influenzladungen immer nur sehr nahe an der Oberfläche, sind aber keine echten Oberflächenladungen.
\par
Da die elektrischen Feldlinien senkrecht auf der Oberfläche stehen, verändert der leitende Körper das elektrische Feld so, dass seine Oberfläche mit einer Äquipotentialfläche übereinstimmt. Zur mathematischen Behandlung wird demzufolge zu dem vorhandenen äußeren elektrischen Feld ein zweites elektrisches Feld eingeführt mit der Randbedingung, dass das superponierte Feld senkrecht auf der Oberfläche steht. Dies ist gleichbedeutend mit der Forderung, dass das elektrische Potential an der Oberfläche überall konstant, der Einfachheit halber gleich 0 ist.
\end{appendices}

\section{Funktionen der Kugelfläche}\label{sphericalfunctions}
Die Kugelflächenfunktionen sind ein vollständiger und orthonormaler Satz von Eigenfunktionen des Winkelanteils des Laplace-Operators. Dieser Winkelanteil zeigt sich, wenn der Laplace-Operator in Kugelkoordinaten geschrieben wird. Die Kugelflächenfunktionen haben eine große Bedeutung für die Lösung partieller Differentialgleichungen. Sie treten zum Beispiel bei der Berechnung von Atomorbitalen auf, da die beschreibende zeitunabhängige Schrödingergleichung den Laplace-Operator enthält und sich das Problem am besten in Kugelkoordinaten lösen lässt.\hfill\break
\\
In Kugelkoordinaten $(r,\theta,\varphi)$ lässt sich der Laplace-Operator wie folgt schreiben:
\[\Delta=\frac{1}{r^2}\frac{\partial}{\partial r}\left(r^2\frac{\partial}{\partial r}\right)+\frac{1}{r^2}\Delta_{\theta,\vphi}\]
\[\Delta_{\theta,\vphi}=\frac{1}{\sin\theta}\frac{\partial}{\partial\theta}\sin\theta\frac{\partial}{\partial\theta}+\frac{1}{\sin^2\theta}\frac{\partial^2}{\partial\vphi^2}\]
\noindent

Die Eigenfunktionen der Operatoren $\Delta_{\theta,\varphi}$ und $i\frac{\partial}{\partial\varphi}$,
\[\Delta_{\theta,\vphi}Y_{lm}(\theta,\vphi)=-l(l+1)Y_{lm}(\theta,\vphi)\]
\[i\frac{\partial}{\partial\vphi}Y_{lm}(\theta,\vphi)=-mY_{lm}(\theta,\vphi)\]

heißen \textbf{Kugelflächenfunktionen} $Y_{lm}(\theta,\vphi)$:
\[Y_{lm}(\theta,\vphi);\ l=0,1,2,\ldots,\ m=-l,-l+1,\ldots,l-1,l\]
\[Y_{lm}(\theta,\vphi)=\sqrt{\frac{2l+1}{4\pi}\frac{(l-m)!}{(l+m)!}}P_l^m(\cos\theta)\exp(im\vphi)\]
\[Y_{l-m}(\theta,\vphi)=(-1)^mY_{lm}^*(\theta,\vphi)\]

\noindent
$P_l^m(z)$ sind \textbf{zugeordnete Legendre-Polynome}
\[P_l^m(z)=(-1)^m(1-z^2)^{\nicefrac{m}{2}}\frac{\mathrm{d}^m}{\mathrm{d}z^m}P_l(z)\]
\[P_l^{-m}(z)=(-1)^m\frac{(l-m)!}{(l+m)!}P_l^m(z)\]

welche die Lösungen der  \textbf{verallgemeinerten Legendre-Gleichung} sind:
\[\frac{\dd}{\dd z}\left[(1-z^2)\frac{\dd P}{\dd z}\right]+\left[l(l+1)-\frac{m^2}{1-z^2}\right]P(z)=0\]

\noindent
$P_l(z)$ sind \textbf{Legendre-Polynome}
\[P_l(z)=\frac{1}{2^ll!}\frac{\dd^l}{\dd z^l}(z^2-1)^l\]

welche die Lösungen der  \textbf{gewöhnlichen Legendre-Gleichung} sind:
\[\frac{\dd}{\dd z}\left[(1-z^2)\frac{\dd P}{\dd z}\right]+l(l+1)P(z)=0\]
\[P_l(\pm1)=(\pm1)^l\qquad\rlap{\scriptsize{nicht auf 1 normiert}}\]

\noindent
Orthogonalitätsrelationen:
\[\int_{-1}^{+1}\dd zP_l(z)P_k(z)=\frac{2}{2l+1}\delta_{lk}\]
\[\int_{-1}^{+1}\dd zP_l^m(z)P_k^m(z)=\frac{2}{2l+1}\frac{(l+m)!}{(l-m)!}\delta_{lk}\]
\[\int_0^{2\pi}\dd\vphi \mathrm{e}^{i(m-m'\vphi)}=2\pi\delta_{mm'}\]
\[\int_0^{2\pi}\dd\varphi\int_{-1}^{+1}\dd\cos\theta Y_{l'm'}^\star(\theta,\vphi)Y_{lm}(\theta\vphi)=\delta_{ll'}\delta_{mm'}\]

\noindent
Vollständigkeitsrelationen:
\[\frac{1}{2}\sum_{l=0}^\infty(2l+1)P_l(z')P_l(z)=\delta(z-z')\]
\[\sum_{l=0}^{\infty}\sum_{m=-l}^{+l}Y_{lm}^*(\theta',\varphi')Y_{lm}(\theta,\varphi)=\delta(\varphi-\varphi')\delta(cos\theta-cos\theta')\]

\noindent
Entwicklungssatz:
\[f(\vec{r})=f(r,\theta,\varphi)=\sum_{l=0}^{\infty}\sum_{m=-l}^{+l}R_{lm}(r)Y_{lm}(\theta,\varphi)\]
\[R_{lm}(r)=\int_0^{2\pi}\dd\varphi\int_{-1}^{+1}\dd \cos\theta f(r,\theta,\varphi)Y_{lm}^*(\theta,\varphi)\]

\noindent
Additionstheorem:
\[\sum_{m=-l}^{+l}Y_{lm}^*(\theta',\varphi')Y_{lm}(\theta,\varphi)=\frac{2l+1}{4\pi}P_l(\cos\gamma)\ ,\qquad\gamma=\sphericalangle(\theta'\varphi',\theta\varphi)\]

\noindent
Spezielle Funktionen:
\begin{align*}
P_0(z)\,&=\,1\\
P_1(z)\,&=\,z\\
P_2(z)\,&=\,\frac{1}{2}(3z^2-1)\\
P_3(z)\,&=\,\frac{1}{2}(5z^3-3z)\ ,\dots;\\
Y_{00}\,&=\,\frac{1}{\sqrt{4\pi}}\\
Y_{11}\,&=\,-\sqrt{\frac{3}{8\pi}}\sin\theta e^{i\varphi}\\
Y_{10}\,&=\,\sqrt{\frac{3}{4\pi}}\cos\theta\\
Y_{22}\,&=\,\frac{1}{4}\sqrt{\frac{15}{2\pi}}\sin^2\theta e^{2i\varphi}\\
Y_{21}\,&=\,-\sqrt{\frac{15}{8\pi}}\sin\theta\cos\theta e^{i\varphi}\\
Y_{20}\,&=\,\sqrt{\frac{5}{4\pi}}\Bigg(\frac{3}{2}\cos^2\theta-\frac{1}{2}\Bigg)\ ,\dots;
\end{align*}

\begin{thebibliography}{9}
\bibitem{homkugelpot}
OnlineMathe das mathe-forum
\textit{Elektrisches Potential einer homogen geladenen Kugel}
\\\texttt{https://www.onlinemathe.de/forum/Potential-einer-homogen-geladenen-Kugel}
\end{thebibliography}

\chapter{Hausaufgaben}

\section{Blatt 10}

\subsection{Nummer 1}

Exponentialansatz

\[\phi_E=\pi r^2 E\]
\[E=\frac{U}{d}\]
\[\vec{\nabla}\times\vec{B}=\mu_0\vec{j}+\vepsilon_0\mu_0\dot{\vec{E}}\]

\subsection{Nummer 2}

$\vec{B}$-Feld eines Dipols:
\[\vec{B}=\frac{\mu_0}{4\pi}\left[\frac{3\vec{r}(\vec{m}\cdot\vec{r})}{r^5}-\frac{\vec{m}}{r^3}\right]\]

\[\vec{\nabla}\times\vec{E}=-\frac{\partial\vec{B}}{\partial t}\]
\[U_\mathrm{ind}=-\frac{\dd}{\dd t}\int_F\dd\vec{f}\cdot\vec{B}\]

\subsection{Nummer 3}

\url{https://en.wikipedia.org/wiki/Maxwell's_equations#Macroscopic_formulation}

\subsection{Nummer 4}

Coulomb-Eichung:
\[\vec{\nabla}\cdot\vec{A}=0\]

Lorentz-Eichung:
\[\vec{\nabla}\cdot\vec{A}+\vepsilon\mu\frac{\partial\phi}{\partial t}=0\]

\section{Blatt 12}

\subsection{Nummer 1}
Maxwell:
\[\vabla\cdot\vec{E}=\frac{\rho}{\epsilon}\xLeftrightarrow{Vakum}\vabla\cdot\vec{E}=0\]
\[\vabla\times\vec{E}=-\frac{\partial\vec{B}}{\partial t}\]

\[k=\frac{2\pi}{\lambda}\]

Poynting:
\[\vec{S}=\vec{E}\times\vec{B}\]

mittlere Energiedichte:
\[w_\mathrm{em}=\frac{1}{2}(E^2+B^2)\]

In der Aufgabe gegeben:\\
Gesamtenergie der Welle
\[W=\epsilon_0\int\dd^3r|E_0(x,y)|^2\]
Drehimpuls
\[\vec{L}=\int\dd^3r\vec{r}\times\bar{\vec{S}}\]

\subsection{Nummer 2}

\[|e^{ix}|=1\]
Maxwell:
\[vabla\times\vec{B}=\vec{J}+\underbrace{\epsilon_0\mu_0}_{\frac{1}{c^2}}\frac{\partial\vec{E}}{\partial t}\]
Leistungsdichte:
\[\vec{j}=\sigma\vec{E}\]

\subsection{Nummer 3}

Rechnung bezieht sich auf Formeln vom ersten Blatt.
Greensche Funktion des Laplace-Operators:
\[\Delta_r\frac{-1}{4\pi}\frac{1}{|\vec{r}-\vec{r}'|}=\delta^3(\vec{r}-\vec{r}')\]

\chapter{Skript}

\section{Elektrostatik}

\subsection{Elektrische Ladung und Coulomb'sches Gesetz}

\[Q=\int_V\dd^3r\rho(\vec{r})\]

\[\vec{F}_1=k\sum_{j=2}^N\frac{q_1q_j}{r^2_{1j}\vec{e}_{r1j}}\]

\[k=\frac{1}{4\pi\vepsilon_0}\]

\subsection{Elektrisches Feld}

Punktladung:
$\vec{F}=q\vec{E}\Rightarrow$
\[\vec{E}(\vec{r})=\frac{1}{4\pi\vepsilon_0}\sum_{j=1}^Nq_j\frac{\vec{r}-\vec{r}_j}{|\vec{r}-\vec{r}_j|^3}\]
\[\rho(\vec{r})=q\delta(\vec{r}-\vec{r}_0)\]
\[\int_V\dd^3r\rho(\vec{r})=q\]

Kontinuierliche Ladungsverteilung:
\[\vec{E}(\vec{r})=\frac{1}{4\pi\vepsilon_0}\int_V\dd^3r'\rho(\vec{r}')\frac{\vec{r}-\vec{r}'}{|\vec{r}-\vec{r}'|^3}\]

%page 8


















\end{document}
