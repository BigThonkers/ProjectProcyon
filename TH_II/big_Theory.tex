\PassOptionsToPackage{dvipsnames}{xcolor}
\documentclass[titlepage,11pt,a4paper,ngerman]{report}
\usepackage[utf8]{inputenc}
\usepackage[T1]{fontenc}
\usepackage[german]{babel}
\usepackage{graphicx}
\usepackage{wrapfig}
\usepackage{amsmath}
\usepackage{amsfonts}
\usepackage{amssymb}
\usepackage{tikz}
\usepackage{tikz-cd}
\usepackage{nicefrac}
\usepackage{mathtools}
\usepackage{enumerate}
\usepackage{cancel}
\usepackage[hidelinks]{hyperref}
\usepackage{cleveref}
\usepackage{tcolorbox}

\usepackage[margin=1in]{geometry}

\usepackage{placeins}
\usepackage{booktabs}

\usepackage{url}

\usetikzlibrary{arrows}


%Environments und Newcommands:

% Allgemein:

% zu zeigen symbol
\newcommand{\zz}{\fontfamily{cmss} \selectfont{Z\kern-.61em\raise-0.7ex\hbox{Z}:}}
% build over
\newcommand{\bov}[2]{\buildrel{#2} \over{#1}}
% better looking := (defined as)
\newcommand*{\defeq}{\mathrel{\vcenter{\baselineskip0.5ex \lineskiplimit0pt \hbox{\scriptsize.}\hbox{\scriptsize.}}}=}
\newcommand*{\eqdef}{=\mathrel{\vcenter{\baselineskip0.5ex \lineskiplimit0pt \hbox{\scriptsize.}\hbox{\scriptsize.}}}}


\newcommand{\hfw}{\color{RubineRed}\tx{ $\star$hier fehlt was$\star$ } \color{black}}
\def\checkmark{\tikz\fill[scale=0.4](0,.35) -- (.25,0) -- (1,.7) -- (.25,.15) -- cycle;} 

%Text:
\newcommand{\tx}[1]{\textrm{#1}}
\newcommand{\const}{\tx{const.}}

\newcommand{\ul}[1]{\underline{#1}}
\newcommand{\ol}[1]{\overline{#1}}
\newcommand{\ub}[1]{\underbrace{#1}}
\newcommand{\ob}[1]{\overbrace{#1}}

%Mathe:
\newcommand{\verteq}{\rotatebox{90}{$\,=$}}
\newcommand{\equalto}[2]{\underset{\scriptstyle\overset{\mkern4mu\verteq}{#2}}{#1}}
\newcommand{\equaltoup}[2]{\overset{\scriptstyle\underset{\mkern4mu\verteq}{#2}}{#1}}
\newcommand{\custo}[3]{\underset{\scriptstyle\overset{\mkern4mu\rotatebox{-90}{$\,#1$}}{#3}}{#2}}
\newcommand{\custoup}[3]{\overset{\scriptstyle\overset{\mkern4mu\rotatebox{-90}{$\,#1$}}{#3}}{#2}}
\newcommand{\casess}[4]{\left\{ \begin{array}{ll} {#1} & {#2} \\ {#3} & {#4} \end{array} \right.}


%Spezielles:

%Theo:
\newcommand{\lag}{\mathcal{L}}
\newcommand{\ham}{\mathcal{H}}
\newcommand{\prt}[2]{\frac{\partial #1}{\partial #2}}
% new
\newcommand{\eofr}{\vec{E}(\vec{r})}
\newcommand{\pofr}{\Phi(\vec{r})}
\newcommand{\delfkt}{\delta(\vec{r} - \vec{r}_0)}
\newcommand{\gre}{\mathcal G(\vec{r},\vec{r}')}

%LA:
\newenvironment{bew}[1]{\subsection{Bew: #1}}{\hfill$\square$}
\newcommand{\Bew}[2]{\begin{bew}{#1}#2\end{bew}}

\newcommand{\enph}{F: V \to V \textrm{ Endomorphismus}}
\newcommand{\im}{\tx{im}}
\newcommand{\spa}{\tx{span}}
\newcommand{\adj}{\tx{adj}}
\newcommand{\grad}{\tx{grad}}
\newcommand{\ord}{\tx{ord}}

\newcommand{\basis}[3]{\{#1_{#2}, \dots, #1_{#3}\}}
\newcommand{\ska}[2]{\langle #1 , #2 \rangle}
\newcommand{\dmat}[3]{\begin{pmatrix} #1_{#2}&&\\ &\ddots& \\ && #1_{#3} \end{pmatrix}}

%Ex:
\newcommand{\kq}{\frac{1}{4\pi\epsilon_0}}
\newcommand{\uind}{U_{\tx{ind}}}
\newcommand{\folie}[1]{\color{gray}[Folie: #1]\color{black}}

% ANDREZ
\newcommand{\summ}[2]{\sum_{#1}^{#2}}
\newcommand{\intt}[2]{\int_{#1}^{#2}}
\renewcommand{\vec}[1]{\boldsymbol{#1}}
\newcommand{\lcom}[1]{\color{MidnightBlue}#1\color{black}}
\renewcommand{\epsilon}{\varepsilon}
\newcommand{\vabla}{\boldsymbol{\nabla}}
\newcommand{\bei}[1]{\emph{Beispiel:}}
\newcommand{\bem}[1]{\emph{Bemerkung:}}
\renewcommand{\paragraph}[1]{\subsubsection{#1}}

% Boxen:

\tcbuselibrary{theorems}

% mahlt eine box nur um den text mit tittel
\newtcbox{\fribox}[1]{nobeforeafter,colback=white,colframe=red!75!black,fonttitle=\bfseries,title=#1,sharp corners,tcbox raise base}

% mahlt eine große box um alles mit tittel
\newcommand{\frbox}[2]{\begin{tcolorbox}[colback=white,colframe=red!75!black,fonttitle=\bfseries,title=#1]#2\end{tcolorbox}}

% mahlt eine box nur um den text
\newtcbox{\ribox}{nobeforeafter,colback=white,colframe=red!75!black,sharp corners,tcbox raise base}

% mahlt eine große box um alles was drinnen ist
\newcommand{\rbox}[1]{\begin{tcolorbox}[colback=white,colframe=red!75!black]#1\end{tcolorbox}}

% mahlt eine box um mathe innerhalb mathmode
\newcommand{\rmbox}[1]{\tcboxmath[colback=white,colframe=red!75!black]{#1}}

% super box (looks like regular boxed but wraps around anything)
\newenvironment{supbox}{\begin{tcolorbox}[colback=white,colframe=black,sharp corners,boxrule=.5pt]}{\end{tcolorbox}}

\newcommand{\bbb}[2]{\begin{tcolorbox}[colback=white,colframe=black,fonttitle=\bfseries,title=#1,sharp corners,tcbox raise base]#2\end{tcolorbox}}

% array type box with title
\newenvironment{zebox}[1]{\begin{array}{|c|}
		\multicolumn{1}{l}{\tx{#1}} \\
		\hline
		\displaystyle
	}{\\ \hline
\end{array}}

% evtl: \renewcommand{\boxed}{\rmbox}



\hbadness=99999

\begin{document}
	
%\renewcommand{\thechapter}{\Roman{chapter}}

\title{
	{\Huge Theoretische Physik II\\[3pt]Elektrodynamik}\\[1em]
	{\Large Vorlesung von Prof. Dr. Michael Thoss im Wintersemester 2018}}
\author{Andréz Gockel \\ Patrick Munnich \\ Daniil Aktanka}
\date{15.10.2018}
\maketitle
\tableofcontents

\chapter{Mathe}
\section{Grundbegriffe}

\paragraph{Ladung}
\begin{align*}
\llap{\text{diskrete Ladungsverteilung}} \qquad Q &= \sum_{i = 1}^{n} q_i \\
\llap{\text{kontinuierliche Ladungsverteilung}} \qquad Q &= \int_{V} \rho ( \vec{r} ) \; \text{d}^3r \\
\llap{\text{Punktladung}} \qquad \rho ( \vec{r} ) &= q \delta ( \vec{r} - \vec{r}_0)\\
\llap{Poisson-Gleichung}\qquad\Delta\Phi(\vec{r})&=-\frac{1}{\epsilon_0}\rho(\vec{r})\\
\end{align*}

\paragraph{Coulomb’sches Gesetz}
\begin{align*}
\llap{\text{zwei Punktladungen}} \qquad \vec{F}_{12} &= \frac{1}{4 \pi \epsilon_0} \, q_1 q_2 \, \frac{\vec{r}_1 - \vec{r}_2}{\left| \vec{r}_1 - \vec{r}_2\right|^3} = -\vec{F}_{21} \\
\llap{$n$\text{ Punktladungen}} \qquad \vec{F}_1 &= \frac{1}{4 \pi \epsilon_0} \, q_1 \sum_{j=2}^{n} q_j \, \frac{\vec{r} - \vec{r}_j}{\left| \vec{r} - \vec{r}_j\right|^3}
\end{align*}

\paragraph{Elektrisches Feld}
\begin{align*}
\llap{\text{im bel. Raumpunkt}} \qquad \vec{E} (\vec{r}) &= \frac{q}{4 \pi \epsilon_0} \, \frac{\vec{r} - \vec{r}_0}{\left| \vec{r} - \vec{r}_0\right|^3} \\
\llap{\text{diskrete Ladungsverteilung}} \qquad \vec{E}(\vec{r}) &= \frac{1}{4 \pi \epsilon_0} \, \sum_{j=2}^{n} q_j \, \frac{\vec{r} - \vec{r}_j}{\left| \vec{r} - \vec{r}_j\right|^3} \\
\llap{\text{kontinuierliche Ladungsverteilung}} \qquad \vec{E}(\vec{r}) &= \frac{1}{4 \pi \epsilon_0} \int \rho (\vec{r}') \, \frac{\vec{r} - \vec{r}'}{\left| \vec{r} - \vec{r}'\right|^3} \; \text{d}^3r' \\
\llap{\text{Gradientenfeld}} \qquad \vec{E} (\vec{r}) &= \; - \nabla \, \varphi (\vec{r})
\end{align*}

\section{Einführung}
\paragraph{Dirac'sche Delta-Funktion}
\subparagraph{Definition}

\begin{align*}
\int_{V} \delta(\vec{r}-\vec{r}_0) \; \text{d}^3r &\coloneqq
	\begin{cases} 
		1 \, , & r_0 \in V \\
		0 \, , & \text{sonst}
	\end{cases} \\
\delta(\vec{r}-\vec{r}_0) &= 0 \quad \ \forall \: \vec{r} \neq \vec{r_0}
\end{align*}

\emph{Bemerkung}: Die $\delta$-Funktion ist keine Funktion im üblichen mathematischen Sinne. Man bezeichnet sie deshalb als \textbf{uneigentliche Funktion} oder als \textbf{Distribution}. Heuristisch:

\begin{align*}
&\delta(x) = 
	\begin{cases}
		+ \infty \, , & x = 0 \\
		0 \, , & x \neq 0
	\end{cases} \\
&\int_{-\infty}^{\infty} \delta (x) \, dx = 1
\end{align*}

%\[H(x)\coloneqq\int_{-\infty}^x\delta(s)\mathrm{d}s=\begin{cases}0&\textrm{for }x<0\\1&\textrm{for }x\geq0\end{cases}\]

\subparagraph{Formeln}
\begin{align*}
\int_{a}^{b} f(x) \, \delta (x-x_0) \, dx &= 
	\begin{cases}
		f(x_0) \, , & a < x_0 < b \\
		\frac{1}{2} f(x_0) \, , & x_0 = a \lor b \\
		0 \, & \text{sonst}
	\end{cases} \\
`` \ f(x) \, \delta '(x-x_0) &= -f'(x) \, \delta (x-x_0) \ " \qquad \rlap{\text{\scriptsize{(heuristisch)}}} \\
\delta (x - x_0) &= \frac{d}{dx} \Theta (x - x_0) \qquad \rlap{\scriptsize{($ \Theta $ \, \text{sei die Stufenfunktion)}}}
\end{align*} 

\subparagraph{Mehrdimensionale Delta-Funktion}
\begin{align*}
\llap{\text{Kartesisch (x,y,z)}} \qquad \delta(\vec{r} - \vec{r}_0) &= \delta(x - x_0) \, \delta(y - y_0) \, \delta(z - z_0) \\
\llap{\text{Kugel $ (r, \theta, \varphi) $}} \qquad \delta(\vec{r} - \vec{r}_0) &= \frac{1}{r^2_0 \sin\theta_0} \, \delta(r - r_0) \, \delta(\theta - \theta _0) \, \delta(\varphi - \varphi _0) \\
\llap{\text{Zylinder $ (\rho , \phi , z) $}} \qquad \delta(\vec{r} - \vec{r}_0) &= \frac{1}{\rho _0} \, \delta(\rho - \rho _0) \, \delta(\varphi - \varphi _0) \, \delta(z - z _0)
\end{align*}

\paragraph{Produktformeln}
$f, g$ sind skalare Felder, $\vec{F}, \vec{G}$ sind vektor Felder:

\begin{align*}
 	\vabla(fg) &= f \vabla (g) + g \vabla (f)\\
 	\vabla \cdot (f \vec{G}) &= f \vabla \cdot (\vec{G}) + \vec{G} \cdot \vabla (f)\\
 	\vabla (\vec{F} \times \vec{G}) &= \vec{G} \cdot \vabla \times (\vec{F}) - \vec{F} \cdot \vabla \times (\vec{G})\\
 	\vabla \times (f \vec{G}) &= f \vabla \times (\vec{G}) - \vec{G} \times \vabla
\end{align*}

\paragraph{Identitäten}

\begin{align*}
	\vec{\nabla}\times(\vec{\nabla}\vec{F})&=\vec{\nabla}(\vec{\nabla}\cdot\vec{F})-\Delta\vec{F}\\
	\vec{\nabla}\times(\vec{\nabla}f)&=\vec{0}\\
	\vec{\nabla}\cdot(\vec{\nabla}\times\vec{F})&=0\\
	\vec{\nabla}\times(\vec{a}\times\vec{\nabla}f)&=\vec{a}\Delta f-\vec{\nabla}(\vec{a}\cdot\vec{\nabla}f)
\end{align*}

\paragraph{Kugelkoordinaten}

$$
\begin{pmatrix}
	x \\ y\\ z
\end{pmatrix}
=
\begin{pmatrix}
	\rho \sin \phi \cos \theta\\
	\rho \sin \phi \sin \theta\\
	\rho \cos \phi
\end{pmatrix}
, \qquad \int\limits_{-\infty}^{+\infty} f(\vec{r})\ \textrm{d}r = \int\limits_{0}^{2 \pi} \int\limits_{0}^{\pi} \int\limits_{0}^{\infty} f(r,\phi,\theta) r^2 \sin \phi\ \textrm{d}r \textrm{d}\phi \textrm{d}\theta
$$

\paragraph{Gradienten eines skalaren Feldes}

$$\vabla g(\vec{r}) = \sum_i \vec{e}_i \frac{1}{\left|\prt{\vec{r}}{u_i}\right|}\prt{}{u_i}g(\vec{r})$$


\section{Integralsätze}

\paragraph{Gaußscher Satz}
$$\int_{V} \vabla \cdot \vec{E}(\vec{r})\ \mathrm{d}^{3}r = \oint_{F}\vec{E}\ \mathrm{d}\vec{f}$$

\paragraph{Stokes Satz}
$$\int_{F} \vabla \times \vec{a}\ \mathrm{d}\vec{f} = \int_{\partial F}\vec{a}\ \mathrm{d}\vec{r}$$

\paragraph{Greensche Identitäten}

$$\int_{V} \vabla g(\vec{r}) \cdot \vabla h(\vec{r}) + g(\vec{r})\Delta h(\vec{r})\ \mathrm{d}^{3}r = \oint_{\partial V} g(\vec{r}) \vabla h(\vec{r})\ \mathrm{d}\vec{f}$$

$$\int_{V} g(\vec{r}) \Delta h(\vec{r}) g(\vec{r}) - h(\vec{r}) \Delta g(\vec{r})\ \mathrm{d}^{3}r = \oint_{\partial V}g(\vec{r})\vabla h(\vec{r}) - h(\vec{r}) \vabla g(\vec{r})\ \mathrm{d}\vec{f}$$
































\end{document}
